\documentclass[11pt,a4paper]{article}
%\pdfminorversion=5
%\pdfobjcompresslevel=3 
%\pdfcompresslevel=9
\usepackage[utf8]{inputenc}
\usepackage[english]{babel}
\usepackage{fontspec}
\setmainfont{Raleway}

\usepackage[resetlabels]{multibib}
\newcites{book,talk,poster,thesis,journal,reports,pre,exp}{{Series- and conference contributions},{Invited talks and Presentations},{Posters},{Theses},{Journal Papers},{Technical reports},{Preprints},{Raw experimental data}}

\usepackage{fancyhdr}
\usepackage{color}
\usepackage{etoolbox}
\usepackage{a4wide}
\usepackage[hidelinks]{hyperref}

\pagestyle{fancy}
\fancyhf{}
\fancyhead[R]{Dr. Patrick Diehl}
\fancyhead[L]{Courses}
\fancyfoot[C]{(\thepage /\pageref{LastPage})}
\fancyfoot[R]{\today}

\definecolor{color1}{rgb}{0.,0.,0.}% light blue
\definecolor{color2}{rgb}{0.45,0.45,0.45}% dark grey}

\renewcommand{\headrulewidth}{2pt}
\patchcmd{\headrule}{\hrule}{\color{color1}\hrule}{}{}
\renewcommand{\footrulewidth}{1pt}
\patchcmd{\footrule}{\hrule}{\color{color2}\hrule}{}{}


\begin{document}

%%%%%%%%%%%%%%%%%%%%%%%%%%%%%%%%%%%%%%%%%%%%
\section*{Mathematics}
%%%%%%%%%%%%%%%%%%%%%%%%%%%%%%%%%%%%%%%%%%%%

%%%%%%%%%%%%%%%%%%%%%%%%%%%%%%%%%%%%%%%%%%%%
\subsection*{Instructor}
%%%%%%%%%%%%%%%%%%%%%%%%%%%%%%%%%%%%%%%%%%%%

%%%%%%%%%%%%%%%%%%%%%%%%%%%%%%%%%%%%%%%%%%%%
\subsubsection*{Parallel Computations Mathematics (M4997)}
%%%%%%%%%%%%%%%%%%%%%%%%%%%%%%%%%%%%%%%%%%%%
This course will focus on the parallel implementation of computational mathematics problems using modern accelerated C++. The aim of this course is to learn how to quickly write useful efficient C++ programs. The students will not learn low-level C/C++ instead they will learn howto use high-level data structures, iterators, generic strings, and streams (including interactive and file I/O) of the C++ ISO Standard library. In addition, highly-optimized linear algebra libraries are introduced since the course teaches to solve problems, instead of explaining low-level C++ and computer science algorithms, like sorting algorithms, which are provided in the C++ standard library.
\begin{itemize}
\item Taught @LSU: Fall 2019 and Fall 2020
\item \href{https://github.com/diehlpkteaching/ParallelComputationMath}{Slides} and \href{https://github.com/diehlpkteaching/ParallelComputationMathExercise}{Exercises}
\end{itemize}

%%%%%%%%%%%%%%%%%%%%%%%%%%%%%%%%%%%%%%%%%%%%
\subsection*{Teaching assistant}
%%%%%%%%%%%%%%%%%%%%%%%%%%%%%%%%%%%%%%%%%%%%
\begin{itemize}
\item Einf\"uhrung in die Numerische Mathematik (Introduction to numerical mathematics), University of Bonn, 2015
\item Algorithmische Mathematik (Mathematical algorithms), University of Bonn, 2013/2014
\item Wissenschaftliches Rechnen 2 (Scientific Computing 2),  University of Bonn, 2013 
\end{itemize}


%%%%%%%%%%%%%%%%%%%%%%%%%%%%%%%%%%%%%%%%%%%%
\section*{Computer science}
%%%%%%%%%%%%%%%%%%%%%%%%%%%%%%%%%%%%%%%%%%%%

%%%%%%%%%%%%%%%%%%%%%%%%%%%%%%%%%%%%%%%%%%%%
\subsection*{Teaching assistant}
%%%%%%%%%%%%%%%%%%%%%%%%%%%%%%%%%%%%%%%%%%%%
\begin{itemize}
\item Computational Fluid Mechanics, University of Stuttgart, 2012
\end{itemize}

\label{LastPage}             
\end{document}
