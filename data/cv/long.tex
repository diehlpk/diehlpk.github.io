\documentclass[11pt,a4paper,sans]{moderncv}
%\pdfminorversion=5
%\pdfobjcompresslevel=3 
%\pdfcompresslevel=9

\usepackage{fontspec}
\setmainfont{Raleway}

\usepackage{dtk-logos}
% moderncv themes
\moderncvstyle{casual}   
\moderncvcolor{black}                
% adjust the page margins
\usepackage[scale=0.75]{geometry}
%\setlength{\hintscolumnwidth}{3cm}                % if you want to change the width of the column with the dates
%\setlength{\makecvtitlenamewidth}{10cm}           % for the 'classic' style, if you want to force the width allocated to your name and avoid line breaks. be careful though, the length is normally calculated to avoid any overlap with your personal info; use this at your own typographical risks...

% personal data
\name{Dr. Patrick}{Diehl}
\title{Curriculum Vit\ae{}}                               % optional, remove / comment the line if not wanted
\address{Digital Media Center, 340 E Parker Blvd}{Baton Rouge, LA 70803, USA}{}% optional, remove / comment the line if not wanted; the "postcode city" and "country" arguments can be omitted or provided empty
%\phone[mobile]{+49~(176)~32311017}                   % optional, remove / comment the line if not wanted; the optional "type" of the phone can be "mobile" (default), "fixed" or "fax"
\phone[fixed]{+1~(225)~578~5061}
%\phone[fax]{+49~(228)~73~3979}
\email{patrickdiehl@lsu.edu}    
 % optional, remove / comment the line if not wanted
\homepage{www.diehlpk.de}                         % optional, remove / comment the line if not wanted
\social[linkedin]{patrickdiehl}                        % optional, remove / comment the line if not wanted
\social[twitter]{diehlpk}                             % optional, remove / comment the line if not wanted
\social[github]{diehlpk}                              % optional, remove / comment the line if not wanted
%\social[skype]{diehlpk}
\social[orcid]{0000-0003-3922-8419}
%\extrainfo{additional information}                 % optional, remove / comment the line if not wanted
\photo[90pt][0.4pt]{picture}                       % optional, remove / comment the line if not wanted; '64pt' is the height the picture must be resized to, 0.4pt is the thickness of the frame around it (put it to 0pt for no frame) and 'picture' is the name of the picture file
%\quote{Some quote}                                 % optional, remove / comment the line if not wanted


% bibliography with mutiple entries
\usepackage{multibib}
\newcites{book,journal,talk,thesis,reports,exp,pre}{{\underline{Proceedings}},{\underline{Journal articles}},{},{},{\underline{Technical reports}},{},{\underline{Preprints}}}

\usepackage{enumitem}

%----------------------------------------------------------------------------------
%            content
%----------------------------------------------------------------------------------
\begin{document}

\makecvtitle

%%%%%%%%%%%%%%%%%%%%%%%%%%%%%%%%%%%%%%%%%%%%%%%%%%%%%%%%%%%%%%%%%%%%%%%%%%%%%%
\section{Education}
%%%%%%%%%%%%%%%%%%%%%%%%%%%%%%%%%%%%%%%%%%%%%%%%%%%%%%%%%%%%%%%%%%%%%%%%%%%%%%
\cventry{2017}{PhD}{Applied mathematics}{University of Bonn}{Germany}{}
\cventry{2012}{Diploma}{Computer Science}{University of Stuttgart}{Germany}{}
%%%%%%%%%%%%%%%%%%%%%%%%%%%%%%%%%%%%%%%%%%%%%%%%%%%%%%%%%%%%%%%%%%%%%%%%%%%%%%
\section{Awards and Honors}
%%%%%%%%%%%%%%%%%%%%%%%%%%%%%%%%%%%%%%%%%%%%%%%%%%%%%%%%%%%%%%%%%%%%%%%%%%%%%%
\cvitem{2019}{IEEE SCIVIS Contest 2019, First Prize, Visual Analysis of Structure Formation in Cosmic Evolution, \href{https://www.youtube.com/watch?v=ykn3ewqWUcw}{Video}, \href{https://megamol.org/wp-content/uploads/2019/10/poster2-reduced.jpg}{Poster}, and \href{https://ieeexplore.ieee.org/abstract/document/8968855}{Short paper}}
%%%%%%%%%%%%%%%%%%%%%%%%%%%%%%%%%%%%%%%%%%%%%%%%%%%%%%%%%%%%%%%%%%%%%%%%%%%%%%
\section{Grant history}
%%%%%%%%%%%%%%%%%%%%%%%%%%%%%%%%%%%%%%%%%%%%%%%%%%%%%%%%%%%%%%%%%%%%%%%%%%%%%%

%%%%%%%%%%%%%%%%%%%%%%%%%%%%%%%%%%%%%%%%%%%%%%%%%%%%%%%%%%%%%%%%%%%%%%%%%%%%%%
\subsection{Current Research (chronological order; most recent one first)}
%%%%%%%%%%%%%%%%%%%%%%%%%%%%%%%%%%%%%%%%%%%%%%%%%%%%%%%%%%%%%%%%%%%%%%%%%%%%%%
\cvitem{}{
\begin{enumerate}
\item
\begin{description}
\item Grant \#524125 (Hartmut Kaiser)
\item Name of Funding Organization: Pacific Northwest National Laboratory
\item Amount Awarded: \$50,000
\item Period of Grant Award: June 25 - Oct 31 2020
\item Title of Project: High Performance Data Analytics (HPDA) Scalable Second-Order Optimization (SSO)
\item Role on Project: Co-PI
\end{description}
\end{enumerate}
}

%%%%%%%%%%%%%%%%%%%%%%%%%%%%%%%%%%%%%%%%%%%%%%%%%%%%%%%%%%%%%%%%%%%%%%%%%%%%%%
%\subsection{Completed Research (chronological order; most recent one first)}
%%%%%%%%%%%%%%%%%%%%%%%%%%%%%%%%%%%%%%%%%%%%%%%%%%%%%%%%%%%%%%%%%%%%%%%%%%%%%%

%%%%%%%%%%%%%%%%%%%%%%%%%%%%%%%%%%%%%%%%%%%%%%%%%%%%%%%%%%%%%%%%%%%%%%%%%%%%%%
%\section{Allocations}
%%%%%%%%%%%%%%%%%%%%%%%%%%%%%%%%%%%%%%%%%%%%%%%%%%%%%%%%%%%%%%%%%%%%%%%%%%%%%%

%%%%%%%%%%%%%%%%%%%%%%%%%%%%%%%%%%%%%%%%%%%%%%%%%%%%%%%%%%%%%%%%%%%%%%%%%%%%%%
\section{Journal editor}
%%%%%%%%%%%%%%%%%%%%%%%%%%%%%%%%%%%%%%%%%%%%%%%%%%%%%%%%%%%%%%%%%%%%%%%%%%%%%%
\cventry{06/20--current}{Topic editor}{Computational fracture mechanics, Applied mathematics, C++, asynchronous and task-based programming}{The Journal of Open Source Software}{}{}{}
%%%%%%%%%%%%%%%%%%%%%%%%%%%%%%%%%%%%%%%%%%%%%%%%%%%%%%%%%%%%%%%%%%%%%%%%%%%%%%
\section{Research experience}
%%%%%%%%%%%%%%%%%%%%%%%%%%%%%%%%%%%%%%%%%%%%%%%%%%%%%%%%%%%%%%%%%%%%%%%%%%%%%%
\cventry{10/18--current}{Research scientist}{Center for Computation \& Technology}{Louisiana State University}{Baton Rouge, LA, USA}{}
\cvitem{}{
\begin{itemize}
\item Treatment of local boundary conditions in non-local models
\item Review of peridynamic and phase-field models
\item Comparison of phase-field and peridynamic models against experimental data
\item Applying machine learning techniques for auto tuning HPC applications (with Gabriel Laberge)
\end{itemize}
}
\cventry{02/17--09/18}{Postdoctoral fellow}{Laboratory of Multiscale Mechanics}{Polytechnique Montr\'eal}{QC, Canada}{}
\cvitem{}{\begin{itemize}
\item Benchmark peridynamic simulations against experimental data for composite materials
\item Extracting constitutive mechanical parameters in linear elasticity using the virtual fields method within the ordinary state-based peridynamic framework (with Rolland Delorme)
\item  Hybrid image processing approach for crack area detection and tracking using local Digital Image Correlation results (with Ilyass Tabiai)
\end{itemize}}
\cventry{04/13--01/17}{Research Assistant}{Institute for Numerical Simulation}{University Bonn}{Bonn, Germany}{}
\cvitem{}{\begin{itemize}
\item Modeling and simulation of crack and fractures in solids using peridynamic
\end{itemize}}
\cventry{07/12--03/13}{Research Assistant}{Institute for Simulation of large Systems}{University Stuttgart}{Stuttgart, Germany}{}
%%%%%%%%%%%%%%%%%%%%%%%%%%%%%%%%%%%%%%%%%%%%%%%%%%%%%%%%%%%%%%%%%%%%%%%%%%%%%%
\section{Visiting positions}
%%%%%%%%%%%%%%%%%%%%%%%%%%%%%%%%%%%%%%%%%%%%%%%%%%%%%%%%%%%%%%%%%%%%%%%%%%%%%%
\cventry{2015}{Guest Research Assistant}{Center for Computation and Technology}{Louisiana State University}{}{}
%%%%%%%%%%%%%%%%%%%%%%%%%%%%%%%%%%%%%%%%%%%%%%%%%%%%%%%%%%%%%%%%%%%%%%%%%%%%%%
\section{Research Interests}
%%%%%%%%%%%%%%%%%%%%%%%%%%%%%%%%%%%%%%%%%%%%%%%%%%%%%%%%%%%%%%%%%%%%%%%%%%%%%%
\cvitem{Computational engineering}{\begin{itemize}
\item Peridynamics theory for the application in solids, like glassy or composite materials,
\item Validation and verification of simulations against experimental data,
\item Assembly of experimental data for comparison with simulations,
\item Application of machine learning to experiments and simulations.
\end{itemize}}
\cvitem{High Performance Computing}{\begin{itemize}
\item The C\texttt{++} Standard Library for Parallelism and Concurrency (HPX),
\item Asynchronous many task systems and there application in computational engineering.
\end{itemize}}
\cvitem{Open science}{\begin{itemize}
\item Open Source Software for scientific applications,
\item Open data for sharing raw experimental results.
\end{itemize}
}
%%%%%%%%%%%%%%%%%%%%%%%%%%%%%%%%%%%%%%%%%%%%%%%%%%%%%%%%%%%%%%%%%%%%%%%%%%%%%%
\section{Teaching experience}
%%%%%%%%%%%%%%%%%%%%%%%%%%%%%%%%%%%%%%%%%%%%%%%%%%%%%%%%%%%%%%%%%%%%%%%%%%%%%%
\cvitem{Instructor}{
\begin{itemize}
\item Parallel computational mathematics (Math 4997), Louisiana State University, 2019,2020
\end{itemize}
}
\cvitem{Teaching assistant}{
\begin{itemize}
\item Einf\"uhrung in die Numerische Mathematik (Introduction to numerical mathematics), University of Bonn, 2015
\item Algorithmische Mathematik (Mathematical algorithms), University of Bonn, 2013/2014
\item Wissenschaftliches Rechnen 2 (Scientific Computing 2),  University of Bonn, 2013 
\end{itemize}
}

%%%%%%%%%%%%%%%%%%%%%%%%%%%%%%%%%%%%%%%%%%%%%%%%%%%%%%%%%%%%%%%%%%%%%%%%%%%%%%
\section{Certificates}
%%%%%%%%%%%%%%%%%%%%%%%%%%%%%%%%%%%%%%%%%%%%%%%%%%%%%%%%%%%%%%%%%%%%%%%%%%%%%%
\cvitem{Baden--W\"urttemberg Certificate}{Certificate for successful completion of the program in higher education pedagogy by the center for educational development of the state of Baden--W\"urttemberg.}
%%%%%%%%%%%%%%%%%%%%%%%%%%%%%%%%%%%%%%%%%%%%%%%%%%%%%%%%%%%%%%%%%%%%%%%%%%%%%%
\section{Academic-related Professional and Public Service}
%%%%%%%%%%%%%%%%%%%%%%%%%%%%%%%%%%%%%%%%%%%%%%%%%%%%%%%%%%%%%%%%%%%%%%%%%%%%%%
\cvitem{10/17--09/18}{ASSEP Labor relations officers for postdoctoral fellows}
\cvitem{03/20--current}{Liaison for the Louisiana district of the SIAM Texas-Louisiana Section\linebreak
Duties:
\begin{itemize}
\item Making sure that people at universities, research institutions and industry in your district know
  about our activities and getting their suggestions on what we can do better
\item Serving on the organizing committee for the annual meeting
\end{itemize}
}
%%%%%%%%%%%%%%%%%%%%%%%%%%%%%%%%%%%%%%%%%%%%%%%%%%%%%%%%%%%%%%%%%%%%%%%%%%%%%%
\section{Organization of Conferences, Workshops and Symposia}
%%%%%%%%%%%%%%%%%%%%%%%%%%%%%%%%%%%%%%%%%%%%%%%%%%%%%%%%%%%%%%%%%%%%%%%%%%%%%%
\cvitem{Symposia}{\begin{itemize}
\item Modeling and Simulation for Complex Material Behavior, 14th U.S. National Congress on Computational Mechanics, \href{http://14.usnccm.org/MS402}{Link}.
\item Peridynamic Theory and Multiscale Methods for Complex Material Behavior, 14th World Congress on Computational Mechanics (WCCM XIV).
\item Peridynamic Theory and Multiscale Methods for Complex Material Behavior, 16th National Congress on Computational Mechanics. 
\end{itemize}}
\cvitem{Workshops}{\begin{itemize}
\item Workshop on Experimental and Computational Fracture Mechanics:  Validating peridynamics and phase field models for fracture
prediction and experimental design, \href{http://wfm2020.usacm.org/}{Link}. Sponsored by
\begin{itemize}
\item US Association for Computational Mechanics,
\item Center for Computation \& Technology at Louisiana State University, 
\item Oak Ridge National Laboratory,
\item Society for Experimental Mechanics,
\item U.S. National Committee on Theoretical and Applied Mechanics (AmeriMech)
\end{itemize}
\end{itemize}}

\cvitem{Panel}{
\begin{itemize}
\item TBAA: Task-Based Algorithms and Applications, Moderator, "International Conference for High Performance Computing, Networking, Storage and Analysis (SC)" 2020
\end{itemize}
}
%%%%%%%%%%%%%%%%%%%%%%%%%%%%%%%%%%%%%%%%%%%%%%%%%%%%%%%%%%%%%%%%%%%%%%%%%%%%%%
\section{Conference and Workshop Grants}
%%%%%%%%%%%%%%%%%%%%%%%%%%%%%%%%%%%%%%%%%%%%%%%%%%%%%%%%%%%%%%%%%%%%%%%%%%%%%%
\cvitem{2020}{\textbf{AmeriMech symposium}: Experimental and Computational Fracture Mechanics: Validating peridynamics and phase field models for fracture prediction and experimental design (\$4000)}

%%%%%%%%%%%%%%%%%%%%%%%%%%%%%%%%%%%%%%%%%%%%%%%%%%%%%%%%%%%%%%%%%%%%%%%%%%%%%%
\section{Advising and related student services}
%%%%%%%%%%%%%%%%%%%%%%%%%%%%%%%%%%%%%%%%%%%%%%%%%%%%%%%%%%%%%%%%%%%%%%%%%%%%%%
\cvitem{Co-supervised theses}{
\begin{itemize}
\item Pfander, David: Eine künstliche Intelligenz für das Kartenspiel Tichu, Studienarbeit Nr. 2398, 2013.
\item Kanis, Sebastian: GPU-based Numerical Integration in the Partition of Unity Method, Diplomarbeit Nr. 3405, 2013.
\end{itemize}
}
\cvitem{Graduate Committee Member}{
\begin{itemize}
\item Master thesis: Maxwell Reeser
\end{itemize}
}
%%%%%%%%%%%%%%%%%%%%%%%%%%%%%%%%%%%%%%%%%%%%%%%%%%%%%%%%%%%%%%%%%%%%%%%%%%%%%%
\section{Publications}
%%%%%%%%%%%%%%%%%%%%%%%%%%%%%%%%%%%%%%%%%%%%%%%%%%%%%%%%%%%%%%%%%%%%%%%%%%%%%%
\nocitejournal{diehl2021octo,10.1093/mnras/stab937,Diehl2016FRAC,busler2017,diehl2018long,Diehl2019survey,TABIAI2019106485,DaiB:2019:PDS:3295500.3356221,Delorme2020,Kaiser2020,PRUDHOMME2020113391,diehl2018implementation,9405442,Jha2021,https://doi.org/10.1002/cpe.6893} 

\bibliographystylejournal{plain}
\bibliographyjournal{../list/publications}  
\nocitebook{Diehl.Schweitzer:2014,Franzelin.Diehl.Pfluger:2014,Diehl.Schweitzer:2015,heller2016,Diehl2017a,Heller2017,diehl2018integration,tohid2018asynchronous,Zhang:2019:IHM:3318170.3318191}
\bibliographystylebook{plain}
\bibliographybook{../list/publications}     
\nocitereports{Diehl.Lipton.Schweitzer:2016,zhang2020supporting}
\bibliographystylereports{plain}
\bibliographyreports{../list/publications}          
\nocitepre{marcello2021octotiger,diehl2021performance}

\bibliographystylepre{plain}
\bibliographypre{../list/publications}         
%%%%%%%%%%%%%%%%%%%%%%%%%%%%%%%%%%%%%%%%%%%%%%%%%%%%%%%%%%%%%%%%%%%%%%%%%%%%%%
\section{Invited talks and Presentations}
\nocitetalk{Diehl:2014,Diehl:2014*3,Diehl:2013,Diehl:2014*1,Diehl:2014*2,Diehl:2013*2,Diehl:2013*1,Diehl:2015*2,Diehl:2016SIAM,Diehl:2016-mafleap,Diehl:2016-wccm,Diehl:2015,Diehl:2016-sandiego,Diehl:2017-him,Diehl2017MontrealCanada,Diehl2017TexasAustin,Diehl:2018*2,Diehl:2018*3,Diehl:2018-wccm,Diehl:2018-sc,Diehl:2019-emi}
\bibliographystyletalk{plain}
\bibliographytalk{../list/publications}
%%%%%%%%%%%%%%%%%%%%%%%%%%%%%%%%%%%%%%%%%%%%%%%%%%%%%%%%%%%%%%%%%%%%%%%%%%%%%%
\section{Thesis}
\nocitethesis{Diehl2017,Diehl:2012}
\bibliographystylethesis{plain}
\bibliographythesis{../list/publications} 
%%%%%%%%%%%%%%%%%%%%%%%%%%%%%%%%%%%%%%%%%%%%%%%%%%%%%%%%%%%%%%%%%%%%%%%%%%%%%%
\section{Raw experimental data}
\nociteexp{ilyass_tabiai_2018_1172068}
\bibliographystyleexp{plain}
\bibliographyexp{../list/publications} 
%%%%%%%%%%%%%%%%%%%%%%%%%%%%%%%%%%%%%%%%%%%%%%%%%%%%%%%%%%%%%%%%%%%%%%%%%%%%%%
\section{Professional Organizations}
%%%%%%%%%%%%%%%%%%%%%%%%%%%%%%%%%%%%%%%%%%%%%%%%%%%%%%%%%%%%%%%%%%%%%%%%%%%%%%
\cvitem{}{
\begin{itemize}
\item Society for Industrial and Applied Mathematics (SIAM)
\item Association for Computing Machinery (ACM)
\item Informatik-Forum Stuttgart e. V.
\end{itemize}
}
%%%%%%%%%%%%%%%%%%%%%%%%%%%%%%%%%%%%%%%%%%%%%%%%%%%%%%%%%%%%%%%%%%%%%%%%%%%%%%
\section{Reviewer}
%%%%%%%%%%%%%%%%%%%%%%%%%%%%%%%%%%%%%%%%%%%%%%%%%%%%%%%%%%%%%%%%%%%%%%%%%%%%%%
\cvitem{}{International Journal of Mechanical Sciences, Fatigue \& Fracture of Engineering Materials \& Structures, Computer Physics Communications, International Journal of Fracture, Parallel Computing,International Journal of High Performance Computing Applications, Computer Methods in Applied Mechanics and Engineering.}
%%%%%%%%%%%%%%%%%%%%%%%%%%%%%%%%%%%%%%%%%%%%%%%%%%%%%%%%%%%%%%%%%%%%%%%%%%%%%%
\section{References}
%%%%%%%%%%%%%%%%%%%%%%%%%%%%%%%%%%%%%%%%%%%%%%%%%%%%%%%%%%%%%%%%%%%%%%%%%%%%%%
\cvitem{Phd advisers}{\begin{itemize}
\item Dr.~Marc~Alexander~Schweitzer, Institute for Numerical Simulation, University of Bonn, Germany, E-Mail: schweitzer@ins.uni-bonn.de
\item Dr.~Daniel~Peterseim, Numerische Mathematik, University of Augsburg, Germany, E-Mail: daniel.peterseim@math.uni-augsburg.de
\end{itemize}}
\cvitem{Postdoctoral fellow adviser}{\begin{itemize}
\item Dr.~Serge~Prudhomme, Department of Mathematical and Industrial Engineering, Polytechnique Montr\'eal, Canada, E-Mail: serge.prudhomme@polymtl.ca
\end{itemize}}
\cvitem{Collaborators}{\begin{itemize}
\item Dr.~Robert~Lipton, Mathematics of Materials Science, Louisiana State University, USA, E-Mail: lipton@math.lsu.edu
\item Dr.~Thomas~Ertl, Visualization and Interactive Systems Institute, University of Stuttgart, Germany, E-Mail: thomas.ertl@vis.uni-stuttgart.de
\item Dr.~Hartmut~Kaiser, Department of Computer Science, Louisiana State University, USA, E-Mail: hkaiser@cct.lsu.edu
\item Dr.~Juhan~Frank, Department of Physics \& Astronomy, Louisiana State University, USA, E-Mail: frank@rouge.phys.lsu.edu
\item Dr.~Steven~R.~Brandt, Center of Computation \& Technology, Louisiana State University, USA, E-Mail: sbrandt@cct.lsu.edu
\item Dr.~Ilyass~Tabiai, École de Technologie Supérieur, Canada, E-Mail: ilyass.tabiai@etsmtl.net
\end{itemize}}
\begin{center}
Last update: \today
\end{center}
\end{document}


%% end of file `template.tex'.
